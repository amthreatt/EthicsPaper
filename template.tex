\documentclass[10pt,twocolumn]{article} 

\usepackage{oxycomps} % use the main oxycomps style file


\bibliography{references4}

\pdfinfo{
    /Title (Writing Your Oxy CS Comps Paper in LaTeX)
    /Author (Justin Li)
}

\title{Computer Science Ethics Paper}

\author{Amelia Threatt}
\affiliation{Occidental College}
\email{athreatt@oxy.edu}

\begin{document}

\maketitle

\begin{abstract}
For the class of 2023 comps requirement for the Computer Science Major, I have decided to construct my own automated online math tutoring machine. The math tutoring machine will be a website designed to take in single variable algebraic equations and provide the user with guided prompts to help students solve for a variable, while simultaneously checking their work at each step. Assistive technology (AT) has been used to help students with learning disabilities, by providing a helpful learning tool that can be utilized in the classroom and out of the classroom. The difference between my website and typical assistive technology is my website will be designed for anyone who needs additional math help whereas assistive technology is marketed towards and designed specifically for students with learning disabilities. 
\end{abstract}

\section{Introduction}
 In the construction of an automated online math tutoring machine, there are ways to address some ethical concerns however it is not possible to provide a solution to all concerns on the one website. This is because when addressing one learning disability the implications could be negative for a different learning disability, other than learning disabilities educational websites are not optimal for accessibility, given the scope of this website only being able to solve algebraic problems this tool will concentrate the power to a small group of math students. 

Intelligent tutoring systems can be helpful in increasing the accessibility of tutoring to more people, especially if the website is free which mine will be. The website being free will remove any financial burden that may come with hiring a tutor. Utilizing the web platform will also allow the student to access the information and help at any time, and from anywhere which may be helpful for students with mobility issues. Although these systems can be helpful to some it does come with some pitfalls of being unable to provide everyone with everything they need for the website to be most successful. 

There are instances where an online intelligent tutoring system is not going to be beneficial for some students, due to various accessibility issues, cognitive and learning disorders, and physical impairments. At this point and time not everyone has access to the internet, and even if they do not everyone is knowledgeable about the internet and how some devices work. Websites inherently are not completely accessible.

Due to the lack of accessibility within the internet, the website will be monopolizing power to people who have access to the internet which is not an equally diverse group of people, along with that the website will be most easily accessible to those with computer literacy and are able to navigate the website from a compatible device. 

In an attempt to create a fully ethical website the designer should take into account all forms of where accessibility will be challenged and create a solution, however, a solution for one problem could create an entirely new problem for another people. To an extent the accessibility features can be user-customized but at what point will all forms of inaccessibility be challenged begging the question of technological solutionism. At the end of the day, in-person tutors came first, intelligent tutoring systems came afterward, if there a way to design a website full capable of the personalization an in-person tutor could provide?

\section{Argument}
In this section I will be breaking down the various ways in which this website is not completely accessible to all, and therefore not perfectly ethical.
\subsection{Internet Accessibility}

In 2019 the National Assessment of Educational Progress reported that no state in the United States of America had 100 percent internet and computer or tablet access at home for fourth or eighth graders \cite{hemphill_nces_nodate}. However, suppose from 2019 to now every student got a computer or even just access to the internet did change and every student has received a computer or tablet and internet access in between that time, that would only allow them a year or two to work on the computer literacy which may not be sufficient time to be completely competent in working a computer and navigating the internet. So aside from the website design, there is already a percentage of students who do not have access to the internet and will not be able to utilize the online teaching tool. To bypass this inaccessibility as much as possible the website would need to be compatible with computers, tablets, and phones. Increasing the number of devices the website can function well on increases the accessibility of the website overall. As seen in the U.S. Department of Commerce Census Bureau it is shown that post-pandemic internet accessibility did increase, however, there is a significant disparity between socioeconomic statuses  \cite{irwin_nces_nodate}. Nonetheless what was made apparent is if a student does not have access to the internet through a computer or tablet there is a good chance that they do have internet access via a smartphone. So the ultimate solution to access my website, or my goal in creating an intelligent tutoring system would be to make sure the website platform is easily usable via a search engine on a smartphone.


\subsection{Mental and Developmental Disabilities}
The website will be using the Orton Gillingham methodology of teaching online, this teaching style was designed for students who struggle with ADHD, ADD, Dyslexia, and others. It was designed for students with learning disabilities, to say the least, it emphasizes a focus on providing a “direct, explicit, multisenory, structured, sequential, diagnostic, and perspective way to teach” \cite{ahearn_what_2016}\cite{noauthor_is_nodate}. The Orton Gillingham methodology has been used before with English and has recently been adapted for mathematics teaching, and using it as a basis for the construction of the website will make it algorithmically compatible for some students who need the extra clarity in intelligent tutoring systems. The emphasis on clear step-by-step instructions is beneficial for students who may be diagnosed with learning disabilities, and also helpful for those who may not have had the opportunity to get a diagnosis. By advertising the website as a tool for all students as opposed to just students with learning disabilities everyone will get the benefits that the Orton Gillingham methodology can provide. However, with that being said this concept of advertising to all contrary to the target audience it was designed for will ultimately diminish the equity in assistive tools. This would result in a power shift from students with learning disabilities getting to use a tool that would help emphasize equity math education for them to students who do not need an assertive tool to aid in bring equity their education and can simply use it to make mathematics “more easy” for them.  

\subsection{Physical Visual and Auditory Disabilities}
A website that requires precise inputs from a keyboard is not optimal for someone with a physical disability such as someone with fine motor skills, or a paraplegic. To combat this issue it would be ideal to implement a text to speech function, however with that then the function would have to account for various accents. In a paper by Adrienne Coyle, they dove into the depths of Speech Recognition: Affordance and Limitations with Accented Speech. The author reported that “studies, and users, have found that speech to text often has a more difficult time processing the speech of people with accents” \cite{coyle_final_nodate}.  An important fact they brought up along with that previous statement is that only a small group of English speakers speak without an accent. So if I were to implement the text to speech functionality in my website to mitigate the inaccessibility for those with physical disabilities inhibiting them from typing there would still be a group of users excluded from this solution due to speech recognition struggling to process English with accents.  

For students with visual impairments, the website may not be able to deliver the same education enhancement for them as it would for s seeing students. The Orton Gillingham methodology has direct and clear instruction, however, it does have an emphasis on multi-sensory experience to help enhance the online teaching experience. In the website design, there are ways to make it easier for a screen ready to decipher and convey the page to the visually impaired user. Typically visually impaired internet users will utilize a screen reader to understand the content on a page, however, if a website is not designed correctly the screen reader can have a hard time processing the page. Designing a webpage more accessible to a screen reader can help when it comes to the design elements of the multi-sensory experience the Orton Gillingham methodology recommends. Other elements the screen reader interpret well are clear heading and subheadings, although this website will not have many of those it will be important to make it clear when the user is supposed to input information and when they are being given instruction for the next step. \cite{noauthor_is_nodate}

The design of the website relies heavily on the visual components to provide clear and concise instruction and the auditory aspect is more in addition to adding to the multisensory experience. With that being said there are still ways in which the website can be adapted to be more ethically enjoyed by auditory impaired students. For elements that utilize sounds as an additional stimulant, there should be a visual marker conveying the same message. 


\section{Counter Argument}
So far we have discussed how this project cannot address all ethical concerns, however, suppose it could, suppose there was a way to create a fully ethical website, what would it look like? Well, firstly for web development to be ethical it should address these three points identified by Adam Scott the author of many books on ethics in web development but one specifically called Building Web Apps For Everyone. The four points he mentions are: we applications should work for everyone, web applications should work everywhere, web applications should respect a user’s privacy and security, and lastly web developers should be considerate of their peers. In this section of counter argument, the focus will be on the website should be able to be used by everyone, and it should be able to be used everywhere seeing as those were two main points of the previous section talking about how the website cannot be fully ethical .\cite{scott_ethical_nodate}

\subsection{Web Applications Should Work for Everyone}
As discussed in earlier sections there are a number of facets to take into consideration when making a web application accessible to all. There are ways to address all the various forms of accessibility for people however it is important to keep in mind the scale of this project. The final product could be a web app that provides readable fonts for people with dyslexia, high contrast colors for those who have vision impairments, auditory emphasis on certain actions, and with that visual graphics in tandem with the auditory outputs of the web app. That is only the beginning, there is so much more that has been mentioned before and even more that has not. There are ways to provide solutions for every issue but for this project, it is helpful to keep the breadth of the project at a manageable place meaning there is a chance that not every solution can be implemented. Therefore this website at this point in time cannot address all the accessibility concerns, meaning it is not completely ethical, however, this does not mean with further development a fully accessible and ethical website cannot be made. There is a clear outline of the target items needed to make an ethical website and with time it can be done. 

\subsection{Web Applications Should Work Everywhere}

As mentioned due to not all students having access to computers, tablets, or the internet it is important to have the website be compatible with all devices. It is said that post-pandemic the number of students with internet access has increased and the likelihood of them having a phone is higher than normal as well, so to make the website accessible to all students regardless of socioeconomic status the website should be built responsibly. The only concern is going to be translating a  lot of the features of learning differences, visual, audio, and physical disabilities onto a smaller device. Aside from that, it would be helpful to implement some sort of offline accessibility, although students may have a device it is not guaranteed they will have constant and stable access to the internet. 

\section{Conclusion}
Websites can be ethically made, there have been papers and books published on the matter. The web development community is seen making efforts to identify where web apps could be ethically improved. The main places where my web app could be improved is in the accessibility aspect, in regards to addressing the needs of various disabilities, and then also will internet accessibility for young students. The target audience of the website is an age group who doesn't have the ability to provide their own technology, so they depend on their guardians and the schools to provide them. However, there are socioeconomic factors at play that do not make this an accessible resource for all students. Not being able to guarantee lower-income students access to this website simply due to the fact that they do not have access to technology is only aiding more in the power divide between them and other students who d have the ability to access the internet. Yet the only way to assuage the inequity would be to regulate who gets to use the website and who does not, and arguably there is no ethical way to choose which students should get access to a free public website ethically.  

It is also important that the push for ethical web development is occurring in real-time, even Facebook is still adding new features to increase accessibility. With that being said what is an ethical web app now may not be an ethical web app ten years from now simply due to the ever-developing nature of ethical computer science work. In this COMPS project as a web developer, it is my duty to identify the pitfalls of my website that can be addressed, and appreciate the fact that the website will never be perfect, but it can always be improved.

\printbibliography 

\end{document}
